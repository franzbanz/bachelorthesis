\addchap*{Kurzzusammenfassung}
\label{kurzzusammenfassung}
{\LARGE Verifikation von Visualisierungen von komplexen Avionik Modellen mit Computer Vision}

Mit der steigenden Komplexit{\"a}t von Anwendungen in der Luftfahrt wird die Nutzung von \acrlong{dsm} (\acrshort{dsm}) in diesem Bereich immer wichtiger. Es erm{\"o}glicht Ingenieuren, effizienter an gr{\"o}{\ss}eren und komplexeren Anwendungen zu arbeiten und reduziert durch automatische Code-Generierung die Anzahl der Fehler in den resultierenden Programmen. Bei sicherheitskritischen Anwendungen jedoch ist \acrshort{dsm} durch die n{\"o}tige Verifikation der Modell-Visualisierungen mit signifikantem Mehraufwand verbunden.\\
Diese Arbeit zielt darauf ab, die Zuverl{\"a}ssigkeit der automatisierten Verifikation von Blockdiagramm-Visualisierungen in \acrshort{dsm} zu verbessern. Techniken der Computer-Vision werden verwendet, um Blockdiagramm-Modelle zu erkennen und zu verarbeiten. Die erkannten Daten werden mit dem urspr{\"u}nglichen Modell verglichen, um Abweichungen zu finden und dem Benutzer innerhalb eines browserbasierten grafischen Modelleditors anzuzeigen.\\
Diese Arbeit erweitert die bestehende Implementierung der Blockdiagramm-Erkennung, um mit komplexen und vielf{\"a}ltigen Diagrammen in drei grafischen \acrlong{dsl}s (\acrshort{dsl}s) zu arbeiten. Die neue Implementierung verwendet eine Kombination von Methoden aus der Computer-Vision, um Kreuzende oder teilweise verdeckte Verbindungslinien in verschiedenen Ausrichtungen, diverse Vertices in unterschiedlichen Gr{\"o}{\ss}en und Anordnungen sowie Textbl{\"o}cke in unterschiedlichen Orientierungen zu erkennen.\\
Diese Verbesserungen demonstrieren das Potenzial von Computer-Vision-Methoden, die Verifikation von \acrshort{dsm}-Modellen zu automatisieren und die Sicherheit in sicherheitskritischen Anwendungen der Luftfahrt zu erh{\"o}hen.\\
Die neue Implementierung wird anhand einer Reihe von 20 Testf{\"a}llen evaluiert, die jeweils ein oder zwei Blockdiagramme mit einem simulierten Fehler enthalten. Die Ergebnisse zeigen, dass die Implementierung in der Lage ist, alle simulierten Fehler korrekt zu identifizieren und textuell anzuzeigen, wobei zwischen verschiedenen Fehlertypen unterschieden wird. Wenn m{\"o}glich, gibt die Implementierung zus{\"a}tzlich zu der textuellen Anzeige auch die Position der gefundenen Fehler im Modelleditor visuell aus.