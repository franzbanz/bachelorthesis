% Page layout
\usepackage{geometry}                    % set page layout
\usepackage{setspace}                    % set line spacing
\usepackage[automark]{scrlayer-scrpage}  % Koma header and footer package

% Language, coding and font
\usepackage{lmodern}                     % Font - as requested by ILS
\usepackage[english]{babel}              % Language setting (last defined is standard)
\usepackage[T1]{fontenc}                 % Hyphenation for words with Umlaute
\usepackage[utf8]{inputenc}              % Coding for correct display of Umlaute
\usepackage[english]{translator}         % Übersetzer
\usepackage{longtable}                   % Tables with page break
\usepackage{tabu}                        % Advanced table package
\usepackage{datetime}                    % Date and time formatting
\usepackage{ltablex}                     % Use this package instead of tabularx for long tables with X columns

% Graphics and colors
\usepackage[pdftex]{graphicx}            % Include graphics
\usepackage{epstopdf}                    % Include eps graphics
\usepackage{color}                       % Colors
\usepackage[svgnames,table]{xcolor}      % Advanced colors
\usepackage{pythonhighlight}             % Highlight Python code
\usepackage{wrapfig}                     % Wrap images in text
\usepackage{framed}                      % Gray background for quotes

% Math
\usepackage{amsmath,amsthm}              % Math environment
\usepackage{textcomp}                     % Degree symbol

% Floats
\usepackage[section]{placeins}           % Control float placement, \FloatBarrier command

% Other
\usepackage[hyphens]{url}                % URL
\usepackage{hyperref}                    % Settings for PDF document
\usepackage{caption}                     % for modification caption format
\usepackage{subcaption}                  % For subfigures
\usepackage{csquotes}                    % Recommended
\usepackage{pdfpages}                    % include PDF pages
\usepackage{lipsum}                      % lorem ipsum blindtext
\usepackage{siunitx}                     % si einheiten
\usepackage{microtype}                   % underfull und overfull box problem minimierung
\usepackage{etoolbox}                    % Appendix Buchstabenseitenzahl
\usepackage{float}                       % In der Lage figures explizit an einer Stelle im Text zu fixieren
\setuptoc{toc}{totoc}                    % Add table of contents to the table of contents
\setuptoc{lof}{totoc}                    % Add list of figures to the table of contents
\setuptoc{lot}{totoc}                    % Add list of tables to the table of contents
\setuptoc{bib}{totoc}                    % Add bibliography to the table of contents
\usepackage{ulem}                        % Underline text
\usepackage{paralist}                    % Modifikation von Listen
\usepackage{titling}                     % \theauthor macro
\usepackage{tabularx}                    % Tabellen

% Customizable Enumerates/Itemizes
\usepackage{enumitem}                    % Bsp.: Option "style=nextline" für eine gleichmäßige Einrückung aller Zeilen

% Tables
\usepackage{lscape}                      % mehrseitige Tabellen
\usepackage{booktabs}                    % \toprule \midrule \bottomrule
\usepackage{colortbl}                    % farbige Tabellen / Tabellen einfärben
\usepackage{multirow}                    % mehrere Zeilen verbinden
\usepackage{array}                       % Hilfsmittel zum Setzen von Tabellen und geordneten Texten im Mathematischem Modus

% Bibliography
\usepackage[
    backend=bibtex,                      % Backends Biblatex
    style=ieee,                          % Bibliogragrafiestil IEEE
    natbib=true                          % Kompatibilitätsmodul natbib
]{biblatex}
\addbibresource{./bibliography/bib.bib}  % Dateipfad zur Bib Datei

% Glossaries
\usepackage[
    xindy,
    nonumberlist,                        % keine Seitenzahlen anzeigen
    nopostdot,                           % keine Punkte
    style=super,                         % Style
    acronym,                             % ein Abkkürzungsverzeichnis erstellen
    toc,                                 % Einträge im Inhaltsverzeichnis
    section=chapter                      % im Inhaltsverzeichnis auf section-Ebene erscheinen
]{glossaries}